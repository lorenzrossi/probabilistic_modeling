% Options for packages loaded elsewhere
\PassOptionsToPackage{unicode}{hyperref}
\PassOptionsToPackage{hyphens}{url}
%
\documentclass[
]{article}
\usepackage{amsmath,amssymb}
\usepackage{lmodern}
\usepackage{iftex}
\ifPDFTeX
  \usepackage[T1]{fontenc}
  \usepackage[utf8]{inputenc}
  \usepackage{textcomp} % provide euro and other symbols
\else % if luatex or xetex
  \usepackage{unicode-math}
  \defaultfontfeatures{Scale=MatchLowercase}
  \defaultfontfeatures[\rmfamily]{Ligatures=TeX,Scale=1}
\fi
% Use upquote if available, for straight quotes in verbatim environments
\IfFileExists{upquote.sty}{\usepackage{upquote}}{}
\IfFileExists{microtype.sty}{% use microtype if available
  \usepackage[]{microtype}
  \UseMicrotypeSet[protrusion]{basicmath} % disable protrusion for tt fonts
}{}
\makeatletter
\@ifundefined{KOMAClassName}{% if non-KOMA class
  \IfFileExists{parskip.sty}{%
    \usepackage{parskip}
  }{% else
    \setlength{\parindent}{0pt}
    \setlength{\parskip}{6pt plus 2pt minus 1pt}}
}{% if KOMA class
  \KOMAoptions{parskip=half}}
\makeatother
\usepackage{xcolor}
\IfFileExists{xurl.sty}{\usepackage{xurl}}{} % add URL line breaks if available
\IfFileExists{bookmark.sty}{\usepackage{bookmark}}{\usepackage{hyperref}}
\hypersetup{
  pdftitle={Probabilistic modeling for causal inference applied to company bankruptcy},
  pdfauthor={Lorenzo Rossi},
  hidelinks,
  pdfcreator={LaTeX via pandoc}}
\urlstyle{same} % disable monospaced font for URLs
\usepackage[margin=1in]{geometry}
\usepackage{graphicx}
\makeatletter
\def\maxwidth{\ifdim\Gin@nat@width>\linewidth\linewidth\else\Gin@nat@width\fi}
\def\maxheight{\ifdim\Gin@nat@height>\textheight\textheight\else\Gin@nat@height\fi}
\makeatother
% Scale images if necessary, so that they will not overflow the page
% margins by default, and it is still possible to overwrite the defaults
% using explicit options in \includegraphics[width, height, ...]{}
\setkeys{Gin}{width=\maxwidth,height=\maxheight,keepaspectratio}
% Set default figure placement to htbp
\makeatletter
\def\fps@figure{htbp}
\makeatother
\setlength{\emergencystretch}{3em} % prevent overfull lines
\providecommand{\tightlist}{%
  \setlength{\itemsep}{0pt}\setlength{\parskip}{0pt}}
\setcounter{secnumdepth}{-\maxdimen} % remove section numbering
\ifLuaTeX
  \usepackage{selnolig}  % disable illegal ligatures
\fi

\title{Probabilistic modeling for causal inference applied to company
bankruptcy}
\author{Lorenzo Rossi}
\date{26/04/2022}

\begin{document}
\maketitle
\begin{abstract}
The paper aims to study the causes of company bankruptcies implementing
probabilistic modeling algorhitms such as mixed graphical models in
order to study causality and to make inference on the factors that lead
to bankruptcy. The conclusion is that Probabilistic modeling techniques
are as performant as other Supervised Learning algorhitm, generally
leading to the same results.
\end{abstract}

\hypertarget{introduction}{%
\section{Introduction}\label{introduction}}

understanding cause-effect relationships between variables and being
able to identify the most important factors in estimating bankruptcies
can yield valuable information. Usually, experimental intervention is
used to find these relationships, as what decides the default of a
company can differ across economic systems, environments and human
actions. As such, this knowledge can be put to further use upon
implementing different models for the actual prediction of bankruptcies.
This paper will focus on analyzing the impact of economic and financial
factors at the firm level on bankruptcy risk. Since it is very unlikely
that only one factor may determine the default of a company, the choice
of a framework in which it is possible to assess the interactions among
variable is preferred. For this purpose probabilistic modeling
algorithms represent a valid choice of framework. Such models models are
used for representing complex domains, conditional independencies and
joint multivariate probability distributions through graphs. The paper
is structured as such: in the first part there will be a brief
literature overview about company bankruptcies and theoretical
background for probabilistic models. The second part is dedicated to the
choice of probabilistic models and the results obtained through
inference on the chosen data set. Finally, the results of this approach
will be confront it with other supervised learning algorithm in order to
evaluate the performances of both approaches.

\hypertarget{dataset}{%
\subsection{Dataset}\label{dataset}}

The dataset comes from the notorious Kaggle dataset
\href{https://www.kaggle.com/datasets/fedesoriano/company-bankruptcy-prediction}{Company
Bankruptcy Prediction: Bankruptcy data from the Taiwan Economic Journal
for the years 1999--2009}. As stated by the authors, data were collected
from the Taiwan Economic Journal for the years 1999 to 2009. Company
bankruptcy was defined based on the business regulations of the Taiwan
Stock Exchange. However, the features selected were 30 out of the
original 95. This was done by hand and taking account of the economic
literature, leading to the selection of a subset of variables. In detai:

Y - Bankrupt?: Class label X1 - ROA(C) before interest and depreciation
before interest: Return On Total Assets(C) X2 - ROA(A) before interest
and \% after tax: Return On Total Assets(A) X3 - ROA(B) before interest
and depreciation after tax: Return On Total Assets(B) X4 - Operating
Gross Margin: Gross Profit/Net Sales X5 - Realized Sales Gross Margin:
Realized Gross Profit/Net Sales X13 - Cash flow rate: Cash Flow from
Operating/Current Liabilities X15 - Tax rate (A): Effective Tax Rate X16
- Net Value Per Share (B): Book Value Per Share(B) X17 - Net Value Per
Share (A): Book Value Per Share(A) X18 - Net Value Per Share (C): Book
Value Per Share(C) X19 - Persistent EPS in the Last Four Seasons:
EPS-Net Income X20 - Cash Flow Per Share X22 - Operating Profit Per
Share (Yuan ¥): Operating Income Per Share X23 - Per Share Net profit
before tax (Yuan ¥): Pretax Income Per Share X37 - Debt ratio \%:
Liability/Total Assets X38 - Net worth/Assets: Equity/Total Assets X42 -
Operating profit/Paid-in capital: Operating Income/Capital X43 - Net
profit before tax/Paid-in capital: Pretax Income/Capital X52 - Operating
profit per person: Operation Income Per Employee X54 - Working Capital
to Total Assets X57 - Cash/Total Assets X59 - Cash/Current Liability X60
- Current Liability to Assets X61 - Operating Funds to Liability X68 -
Retained Earnings to Total Assets X70 - Total expense/Assets X82 - CFO
to Assets X84 - Current Liability to Current Assets X86 - Net Income to
Total Assets X89 - Gross Profit to Sales X95 - Equity to Liability

\hypertarget{literature-overview-for-company-bankruptcies}{%
\section{Literature Overview for company
bankruptcies}\label{literature-overview-for-company-bankruptcies}}

There's a rich literature in microeconomics research about firm
defaults. The aim is to develop the models based on a combination of
these features, and confront them with the results obtained by the
algorithms. The concept of ``failure'' can vary from the narrow
definition of bankruptcy or permanent insolvency to simply
non-achievement of goals (Cochran, 1981; Pretorius, 2009). Altman
constructed a model to predict bankruptcy with a multiple discriminant
analysis finding that profitability, liquidity and solvency were the
most significant factors in predicting bankruptcy (Altman, 1968). Ohlson
explains four different factors he found statistically significant in
affecting the probability of failure: the size of the company, the state
of financial structures, performance and liquidity (Ohlson, 1980).
Various empirical studies (Baldwin et al., 1997) tend to support the
theoretical assumption that firm failure is rarely caused by only one
cause or source. The most common factor in literature and evidence is
that more than one causes or variables are taken into consideration when
investigating bankruptcy.

\hypertarget{theoretical-framework}{%
\section{Theoretical Framework}\label{theoretical-framework}}

The main goal of probabilistic models is to capture conditional
independence relationships between interacting random variables.
Moreover, by being aware of the graph structure of a PGM, one can solve
tasks such as inference.

The starting point is to consider

\[p(x_1, x_2, ..., x_n)\]

as a probability distribution that can be decomposed in

\[p(x_{1j}) = p(x_1)p(x_2|x_1)p(x_3|x_2, x_1)...p(x_j|x_{1j-1})\]

through the chain rule, which expresses the probability of interceptions
through conditional probabilities, much similar to the Bayes rule:

\[P(α|β) = \frac{P(β|α)P(α)}{P(β)}\]

Two events \(α\) and \(β\) are independent given a third event
\(γ (α ⊥⊥ β|γ)\) if \(P(α|β ∩ γ) = P(α|γ)\) and vice-versa
\(P(β|α ∩ γ) = P(β|γ)\)c Which means that knowing \(γ\) makes \(β\)
irrelevant for predicting \(α\) (the same is valid for \(α\) and \(β\)
inverted)

Following these statements and the necessary mathematical iterations,
the Conditional Density Function is met:

\[f_{1|2}(x_1|x_2) = \frac{f_{12}(x_1, x_2)}{f_2(x2)}\] \#\# Mixed
Interactions Models

The dataset in this work is a case where there are both discrete and
continuous variables with 1 discrete variable and 30 continuous
variables. In the literature they are called Mixed Interaction Models,
which are models for qualitative and quantitative variables that combine
log-linear models for discrete variables with graphical Gaussian models
for continuous variables. Moreover, a MIM is called ``Homogeneous'' if
the covariance matrix of the Gaussian variables does not depend on the
values of the discrete variables. In the case for a MIM, the following
density has to be considered:

\[f(i, y) = p(i)(2π)^{−q/2} det(Σ)^{−1/2}exp[-\frac{1}{2}(y − µ(i))^T Σ−1(y − µ(i))\]
which belongs to the exponential family:
\[f(i, y) = exp[g(i) + h(i)^T y −\frac{1}{2}y^T Ky]\] where \(K\) is the
concentration matrix (a simmetric positive \(q × q\) matrix), \(g(i)\),
\(h(i)\) are the log-linear expansion of the probability \(p(i)\)
(Canonical parameters).For a fixed \(i\), \(g(i)\) is a value and
\(h(i)\) is a \(q\) vector. Note that it is possible to impose
conditional independence among variables by setting their interactions
to zero.

The dimension of a MI-model is the number of canonical parameters
composing \(g(i)\) and \(h(i)\) adding to the number of free elements of
the covariance matrix less 1. Under \(M\), \(D\) is asymptotically
\(χ^2\) (k) where k is the difference in dimension between the saturated
model and M. The minimal sufficient statistics for a generator (a, b),
where \(a ⊆ ∆\)and \$b ⊆ Γ \$are: marginal frequency, total of
variables, sub-matrix of the sum of squares matrix.

\hypertarget{probabilistic-modeling-and-results}{%
\subsection{Probabilistic Modeling and
results}\label{probabilistic-modeling-and-results}}

First of all, given the enormous disproportion between the number of
bankruptcies and non-bankruptcies, the number of the former was
increased through the procedure of oversampling performed with the ROSE
package of R, building a database with the same numbero of observation
for the target variable (Bankrupt), with a proportion of 1:1.
Furthermore, the continuous variables were standardized.

\includegraphics{PM_CompanyBankruptcy_files/figure-latex/unnamed-chunk-3-1.pdf}

The analysis with the graphical models was started by inspecting the
relations between continuous variables in the dataset. For this purpose,
the GLASSO model was implemented through the relative \emph{glasso}
package. It gives a fast technique to find the Gaussian graphical model
that maximizes a log-likelihood for \(K\) which is penalized by the
\(L1 − norm |K|\). Note that in this first part of the analysis, the
target variable was not consiedered

After different tests, the model that best matches the interaction of
the variables with respect to economic theory without an extreme
penalization is the model with the value of ρ (the one that penalizes
further connections) equal to 0.4

The model presents a graph with four clusters of variables: the main
cluster collects all those variables that interact with net revenues and
expenses; a second cluster is related only to gross revenues; another
one is made by the variables related to asset values of the company; the
final cluster presents the cash-flow variables. This graph is useful for
understanding the main groups of variables and how the interactions
within them.

\includegraphics{PM_CompanyBankruptcy_files/figure-latex/unnamed-chunk-5-1.pdf}

\includegraphics{PM_CompanyBankruptcy_files/figure-latex/unnamed-chunk-6-1.pdf}

\includegraphics{PM_CompanyBankruptcy_files/figure-latex/unnamed-chunk-7-1.pdf}
The target variable will now be included again in the analysis.

The next two algorithm for graphical representation of the interactions
among variables are the minForest() algorithm from \emph{gRapHD} and
stepw() from \emph{gRim} packages respectively. The first implements the
minForest model which returns the tree or forest that minimizes the
\(-2log-likelihood\), AIC, or BIC. From the direct connections of the
target variable in the plot, it is possible to observe the link between
Tax Rate, Persistent Earning per Share (EPS) and the Net Value of the
assets of the company. The stepw() algorithm perform stepwise selection,
and it shows many more variables to the target. A part from the previous
variables, now it the target variable presents connections with Debt,
Return On Total Assets and Net Profits.

All of these interactions find proof in the economic theory and in the
aforementioned literature.

\includegraphics{PM_CompanyBankruptcy_files/figure-latex/unnamed-chunk-8-1.pdf}
\includegraphics{PM_CompanyBankruptcy_files/figure-latex/unnamed-chunk-8-2.pdf}

Finally, the mgm() algorithm from the its own package was implemented,
using nodewise regression. The k parameter was set equal to three, and a
cross-validation (CV) with ten folds was considered. The result
maintains the same connections as the previous plot but it highlights
four specific variables: Debt Ratio, Net Worth assets, Working Capital
and Cash On Total Assets. Also in this case, the connected variables can
fall into the categories of factors that the literature has shown to be
relevant in predicting company bankruptcy, as they are all linked to
solvency, profitability and liquidity.

\begin{verbatim}
##   |                                                                              |                                                                      |   0%  |                                                                              |--                                                                    |   3%  |                                                                              |-----                                                                 |   6%  |                                                                              |-------                                                               |  10%  |                                                                              |---------                                                             |  13%  |                                                                              |-----------                                                           |  16%  |                                                                              |--------------                                                        |  19%  |                                                                              |----------------                                                      |  23%  |                                                                              |------------------                                                    |  26%  |                                                                              |--------------------                                                  |  29%  |                                                                              |-----------------------                                               |  32%  |                                                                              |-------------------------                                             |  35%  |                                                                              |---------------------------                                           |  39%  |                                                                              |-----------------------------                                         |  42%  |                                                                              |--------------------------------                                      |  45%  |                                                                              |----------------------------------                                    |  48%  |                                                                              |------------------------------------                                  |  52%  |                                                                              |--------------------------------------                                |  55%  |                                                                              |-----------------------------------------                             |  58%  |                                                                              |-------------------------------------------                           |  61%  |                                                                              |---------------------------------------------                         |  65%  |                                                                              |-----------------------------------------------                       |  68%  |                                                                              |--------------------------------------------------                    |  71%  |                                                                              |----------------------------------------------------                  |  74%  |                                                                              |------------------------------------------------------                |  77%  |                                                                              |--------------------------------------------------------              |  81%  |                                                                              |-----------------------------------------------------------           |  84%  |                                                                              |-------------------------------------------------------------         |  87%  |                                                                              |---------------------------------------------------------------       |  90%  |                                                                              |-----------------------------------------------------------------     |  94%  |                                                                              |--------------------------------------------------------------------  |  97%  |                                                                              |----------------------------------------------------------------------| 100%
## Note that the sign of parameter estimates is stored separately; see ?mgm
\end{verbatim}

\includegraphics{PM_CompanyBankruptcy_files/figure-latex/unnamed-chunk-9-1.pdf}

The previous model brought solid results. The next step is to evaluate
numerically the performance of the previous algorithm in terms of
confusion matrix and ROC curve. In the next plots it is possible to
check the number of the false positives and negatives estimated by the
algorithm in the confusion matrix as well as an accuracy score. The MGM
obtained an accuracy score of the 84\%.

\begin{verbatim}
## Confusion Matrix and Statistics
## 
##           Reference
## Prediction    0    1
##          0 3076  366
##          1  715 2662
##                                           
##                Accuracy : 0.8415          
##                  95% CI : (0.8326, 0.8501)
##     No Information Rate : 0.5559          
##     P-Value [Acc > NIR] : < 2.2e-16       
##                                           
##                   Kappa : 0.6826          
##                                           
##  Mcnemar's Test P-Value : < 2.2e-16       
##                                           
##             Sensitivity : 0.8114          
##             Specificity : 0.8791          
##          Pos Pred Value : 0.8937          
##          Neg Pred Value : 0.7883          
##              Prevalence : 0.5559          
##          Detection Rate : 0.4511          
##    Detection Prevalence : 0.5048          
##       Balanced Accuracy : 0.8453          
##                                           
##        'Positive' Class : 0               
## 
\end{verbatim}

\includegraphics{PM_CompanyBankruptcy_files/figure-latex/unnamed-chunk-11-1.pdf}

\includegraphics{PM_CompanyBankruptcy_files/figure-latex/unnamed-chunk-12-1.pdf}

\hypertarget{comparing-other-ml-algorithms}{%
\subsection{Comparing other ML
algorithms}\label{comparing-other-ml-algorithms}}

The MGM algorithm obtained a moderately high accuracy and created fair
graphical representation of the interactions among variables. The next
part of the analysis will be to confront its result with other
``classic'' Machine Learning algorithms. For this comparison were chosen
the following models:

\begin{itemize}
\item
  Logistic Regression
\item
  Random Forest Classifier
\item
  Boosting
\end{itemize}

\hypertarget{logistic-regression}{%
\subsection{Logistic Regression}\label{logistic-regression}}

The regression output reveals the high significance of certain variable
to be the ones highlighted by the graphical model connections. As
expected by economic literature, many of them affect either positively
or negatively the probability of a bankruptcy (i.e.~higher debt risks to
lead to bankruptcy, while higher net worth assets lowers that
possibility etc.).

\begin{verbatim}
## Warning: glm.fit: fitted probabilities numerically 0 or 1 occurred
\end{verbatim}

\begin{verbatim}
## 
## Call:
## glm(formula = Bankrupt. ~ ., family = binomial, data = cb_balanced)
## 
## Deviance Residuals: 
##     Min       1Q   Median       3Q      Max  
## -4.1884  -0.5147  -0.0082   0.5713   2.7188  
## 
## Coefficients:
##                                                          Estimate Std. Error
## (Intercept)                                             -1.994378   0.063044
## ROA.C..before.interest.and.depreciation.before.interest -0.187664   0.035968
## ROA.A..before.interest.and...after.tax                  -0.091707   0.031266
## ROA.B..before.interest.and.depreciation.after.tax       -0.172696   0.033943
## Operating.Gross.Margin                                   0.011738   0.042679
## Realized.Sales.Gross.Margin                              0.039153   0.042537
## Cash.flow.rate                                          -0.036411   0.041412
## Tax.rate..A.                                            -0.050766   0.029401
## Net.Value.Per.Share..B.                                 -0.110229   0.051668
## Net.Value.Per.Share..A.                                 -0.150188   0.052368
## Net.Value.Per.Share..C.                                 -0.163826   0.051240
## Persistent.EPS.in.the.Last.Four.Seasons                 -0.267095   0.045489
## Cash.Flow.Per.Share                                     -0.139830   0.043221
## Operating.Profit.Per.Share..Yuan.Â..                    -0.147633   0.051732
## Per.Share.Net.profit.before.tax..Yuan.Â..               -0.226875   0.047207
## Debt.ratio..                                             0.334091   0.038555
## Net.worth.Assets                                        -0.451235   0.039632
## Operating.profit.Paid.in.capital                        -0.206815   0.052507
## Net.profit.before.tax.Paid.in.capital                   -0.197898   0.046751
## Operating.profit.per.person                             -0.096408   0.041001
## Working.Capital.to.Total.Assets                         -0.248542   0.034814
## Cash.Total.Assets                                       -0.421486   0.047734
## Current.Liability.to.Assets                              0.115648   0.031864
## Operating.Funds.to.Liability                            -0.003648   0.039242
## Retained.Earnings.to.Total.Assets                       -0.040097   0.031685
## Total.expense.Assets                                    -0.076020   0.022847
## CFO.to.Assets                                            0.039336   0.034601
## Current.Liability.to.Current.Assets                      0.085559   0.027614
## Net.Income.to.Total.Assets                              -0.060468   0.026802
## Gross.Profit.to.Sales                                   -0.024991   0.042859
## Equity.to.Liability                                      0.093894   0.027836
##                                                         z value Pr(>|z|)    
## (Intercept)                                             -31.635  < 2e-16 ***
## ROA.C..before.interest.and.depreciation.before.interest  -5.218 1.81e-07 ***
## ROA.A..before.interest.and...after.tax                   -2.933 0.003356 ** 
## ROA.B..before.interest.and.depreciation.after.tax        -5.088 3.62e-07 ***
## Operating.Gross.Margin                                    0.275 0.783297    
## Realized.Sales.Gross.Margin                               0.920 0.357348    
## Cash.flow.rate                                           -0.879 0.379267    
## Tax.rate..A.                                             -1.727 0.084227 .  
## Net.Value.Per.Share..B.                                  -2.133 0.032891 *  
## Net.Value.Per.Share..A.                                  -2.868 0.004131 ** 
## Net.Value.Per.Share..C.                                  -3.197 0.001388 ** 
## Persistent.EPS.in.the.Last.Four.Seasons                  -5.872 4.32e-09 ***
## Cash.Flow.Per.Share                                      -3.235 0.001215 ** 
## Operating.Profit.Per.Share..Yuan.Â..                     -2.854 0.004320 ** 
## Per.Share.Net.profit.before.tax..Yuan.Â..                -4.806 1.54e-06 ***
## Debt.ratio..                                              8.665  < 2e-16 ***
## Net.worth.Assets                                        -11.386  < 2e-16 ***
## Operating.profit.Paid.in.capital                         -3.939 8.19e-05 ***
## Net.profit.before.tax.Paid.in.capital                    -4.233 2.31e-05 ***
## Operating.profit.per.person                              -2.351 0.018706 *  
## Working.Capital.to.Total.Assets                          -7.139 9.39e-13 ***
## Cash.Total.Assets                                        -8.830  < 2e-16 ***
## Current.Liability.to.Assets                               3.629 0.000284 ***
## Operating.Funds.to.Liability                             -0.093 0.925938    
## Retained.Earnings.to.Total.Assets                        -1.265 0.205692    
## Total.expense.Assets                                     -3.327 0.000877 ***
## CFO.to.Assets                                             1.137 0.255609    
## Current.Liability.to.Current.Assets                       3.098 0.001946 ** 
## Net.Income.to.Total.Assets                               -2.256 0.024066 *  
## Gross.Profit.to.Sales                                    -0.583 0.559817    
## Equity.to.Liability                                       3.373 0.000743 ***
## ---
## Signif. codes:  0 '***' 0.001 '**' 0.01 '*' 0.05 '.' 0.1 ' ' 1
## 
## (Dispersion parameter for binomial family taken to be 1)
## 
##     Null deviance: 9452.5  on 6818  degrees of freedom
## Residual deviance: 4995.3  on 6788  degrees of freedom
## AIC: 5057.3
## 
## Number of Fisher Scoring iterations: 6
\end{verbatim}

The following plots show the accuracy of the Logistic model and its
predictions. The model obtain a slightly lower accuracy than the MGM.

\includegraphics{PM_CompanyBankruptcy_files/figure-latex/unnamed-chunk-16-1.pdf}

\begin{verbatim}
## Confusion Matrix and Statistics
## 
##     True
## Pred   0   1
##    0 866 165
##    1 167 847
##                                           
##                Accuracy : 0.8377          
##                  95% CI : (0.8209, 0.8534)
##     No Information Rate : 0.5051          
##     P-Value [Acc > NIR] : <2e-16          
##                                           
##                   Kappa : 0.6753          
##                                           
##  Mcnemar's Test P-Value : 0.9562          
##                                           
##             Sensitivity : 0.8383          
##             Specificity : 0.8370          
##          Pos Pred Value : 0.8400          
##          Neg Pred Value : 0.8353          
##              Prevalence : 0.5051          
##          Detection Rate : 0.4235          
##    Detection Prevalence : 0.5042          
##       Balanced Accuracy : 0.8376          
##                                           
##        'Positive' Class : 0               
## 
\end{verbatim}

\includegraphics{PM_CompanyBankruptcy_files/figure-latex/unnamed-chunk-17-1.pdf}

\hypertarget{random-forest}{%
\subsection{Random Forest}\label{random-forest}}

The RF model was trained using 1000 trees. The plot shows the relevance
of the features in determining the classification task. It is clear that
the first two variables outclass the other for importance in explaining
the classification output. The two are followed by Debt Ratio and Return
On Total Assets. However, most of the relevant variables are not changed
if compared to the previous algorithms. On the contrary, certain ones
maintain constant their high influence on the target variable (i.e.~Net
Worth Assets and Debt Ratio).

\begin{verbatim}
## Warning: `guides(<scale> = FALSE)` is deprecated. Please use `guides(<scale> =
## "none")` instead.
\end{verbatim}

\includegraphics{PM_CompanyBankruptcy_files/figure-latex/unnamed-chunk-19-1.pdf}

The confusion matrix for the RF shows that the model obtains an higher
accuracy (almost 90\%) than the MGM with less wrong prediction.

\begin{verbatim}
## Confusion Matrix and Statistics
## 
##           Reference
## Prediction   0   1
##          0 898  88
##          1 135 924
##                                           
##                Accuracy : 0.891           
##                  95% CI : (0.8766, 0.9041)
##     No Information Rate : 0.5051          
##     P-Value [Acc > NIR] : < 2.2e-16       
##                                           
##                   Kappa : 0.782           
##                                           
##  Mcnemar's Test P-Value : 0.002067        
##                                           
##             Sensitivity : 0.8693          
##             Specificity : 0.9130          
##          Pos Pred Value : 0.9108          
##          Neg Pred Value : 0.8725          
##              Prevalence : 0.5051          
##          Detection Rate : 0.4391          
##    Detection Prevalence : 0.4822          
##       Balanced Accuracy : 0.8912          
##                                           
##        'Positive' Class : 0               
## 
\end{verbatim}

\includegraphics{PM_CompanyBankruptcy_files/figure-latex/unnamed-chunk-20-1.pdf}

\includegraphics{PM_CompanyBankruptcy_files/figure-latex/unnamed-chunk-21-1.pdf}

\#Boosting

Boosting is another approach to improve the predictions resulting from a
decision tree. Like bagging and random forests, it is a general approach
that can be applied to many statistical learning methods for regression
or classification. Boosting builds lots of smaller trees. Unlike random
forests, each new tree in boosting tries to patch up the deficiencies of
the current ensemble. It's a sequential process in which each next model
which is generated is added so as to improve a bit from the previous
model.

The output is similar to the RF: the first three variables remain the
most important in making prediction for for the target variable, while
the other variables in the output show slightly different levels of
importance.

\includegraphics{PM_CompanyBankruptcy_files/figure-latex/unnamed-chunk-23-1.pdf}

\begin{verbatim}
##                                                                                                             var
## Net.worth.Assets                                                                               Net.worth.Assets
## Net.Income.to.Total.Assets                                                           Net.Income.to.Total.Assets
## Debt.ratio..                                                                                       Debt.ratio..
## Cash.Total.Assets                                                                             Cash.Total.Assets
## ROA.B..before.interest.and.depreciation.after.tax             ROA.B..before.interest.and.depreciation.after.tax
## Current.Liability.to.Current.Assets                                         Current.Liability.to.Current.Assets
## Total.expense.Assets                                                                       Total.expense.Assets
## ROA.A..before.interest.and...after.tax                                   ROA.A..before.interest.and...after.tax
## ROA.C..before.interest.and.depreciation.before.interest ROA.C..before.interest.and.depreciation.before.interest
## Operating.Profit.Per.Share..Yuan.Â..                                       Operating.Profit.Per.Share..Yuan.Â..
## Operating.profit.Paid.in.capital                                               Operating.profit.Paid.in.capital
## Retained.Earnings.to.Total.Assets                                             Retained.Earnings.to.Total.Assets
## Persistent.EPS.in.the.Last.Four.Seasons                                 Persistent.EPS.in.the.Last.Four.Seasons
## Net.Value.Per.Share..C.                                                                 Net.Value.Per.Share..C.
## Net.Value.Per.Share..A.                                                                 Net.Value.Per.Share..A.
## Net.Value.Per.Share..B.                                                                 Net.Value.Per.Share..B.
## Net.profit.before.tax.Paid.in.capital                                     Net.profit.before.tax.Paid.in.capital
## Working.Capital.to.Total.Assets                                                 Working.Capital.to.Total.Assets
## Tax.rate..A.                                                                                       Tax.rate..A.
## Operating.profit.per.person                                                         Operating.profit.per.person
## Per.Share.Net.profit.before.tax..Yuan.Â..                             Per.Share.Net.profit.before.tax..Yuan.Â..
## Cash.flow.rate                                                                                   Cash.flow.rate
## Equity.to.Liability                                                                         Equity.to.Liability
## Current.Liability.to.Assets                                                         Current.Liability.to.Assets
## Cash.Flow.Per.Share                                                                         Cash.Flow.Per.Share
## CFO.to.Assets                                                                                     CFO.to.Assets
## Operating.Funds.to.Liability                                                       Operating.Funds.to.Liability
## Operating.Gross.Margin                                                                   Operating.Gross.Margin
## Realized.Sales.Gross.Margin                                                         Realized.Sales.Gross.Margin
## Gross.Profit.to.Sales                                                                     Gross.Profit.to.Sales
##                                                             rel.inf
## Net.worth.Assets                                        19.26205824
## Net.Income.to.Total.Assets                              14.45098592
## Debt.ratio..                                             7.21390807
## Cash.Total.Assets                                        6.49575171
## ROA.B..before.interest.and.depreciation.after.tax        6.37390688
## Current.Liability.to.Current.Assets                      5.55060702
## Total.expense.Assets                                     5.51866230
## ROA.A..before.interest.and...after.tax                   5.43980197
## ROA.C..before.interest.and.depreciation.before.interest  5.28973574
## Operating.Profit.Per.Share..Yuan.Â..                     3.69537791
## Operating.profit.Paid.in.capital                         3.03984054
## Retained.Earnings.to.Total.Assets                        2.21416630
## Persistent.EPS.in.the.Last.Four.Seasons                  2.05142004
## Net.Value.Per.Share..C.                                  1.60513111
## Net.Value.Per.Share..A.                                  1.55766746
## Net.Value.Per.Share..B.                                  1.51436637
## Net.profit.before.tax.Paid.in.capital                    1.36887330
## Working.Capital.to.Total.Assets                          1.36305971
## Tax.rate..A.                                             1.03998436
## Operating.profit.per.person                              0.81853681
## Per.Share.Net.profit.before.tax..Yuan.Â..                0.78854754
## Cash.flow.rate                                           0.72699808
## Equity.to.Liability                                      0.57075199
## Current.Liability.to.Assets                              0.45206189
## Cash.Flow.Per.Share                                      0.43720829
## CFO.to.Assets                                            0.39632424
## Operating.Funds.to.Liability                             0.26898213
## Operating.Gross.Margin                                   0.26026484
## Realized.Sales.Gross.Margin                              0.15271522
## Gross.Profit.to.Sales                                    0.08230403
\end{verbatim}

The accuracy of the Boosting is higher than MGM, but still lower than
the RF.

\begin{verbatim}
## Confusion Matrix and Statistics
## 
##           Reference
## Prediction   0   1
##          0 920 104
##          1 113 908
##                                           
##                Accuracy : 0.8939          
##                  95% CI : (0.8797, 0.9069)
##     No Information Rate : 0.5051          
##     P-Value [Acc > NIR] : <2e-16          
##                                           
##                   Kappa : 0.7878          
##                                           
##  Mcnemar's Test P-Value : 0.5871          
##                                           
##             Sensitivity : 0.8906          
##             Specificity : 0.8972          
##          Pos Pred Value : 0.8984          
##          Neg Pred Value : 0.8893          
##              Prevalence : 0.5051          
##          Detection Rate : 0.4499          
##    Detection Prevalence : 0.5007          
##       Balanced Accuracy : 0.8939          
##                                           
##        'Positive' Class : 0               
## 
\end{verbatim}

\includegraphics{PM_CompanyBankruptcy_files/figure-latex/unnamed-chunk-25-1.pdf}

\includegraphics{PM_CompanyBankruptcy_files/figure-latex/unnamed-chunk-26-1.pdf}

\hypertarget{conclusions}{%
\section{Conclusions}\label{conclusions}}

This paper focused on analyzing the impact of economic and financial
factors that determine company bankruptcies through a probabilistic
modeling framework. The results show that net revenue, asset values,
level of debt, and productivity are the main factors that mostly
determine the occurrence of this event. Algorithms that allow for
graphical representation were implemented in order to find and visualize
the connections and the independencies of the target variable
(Bankruptcy). The MGM model was used to make inference, discover the
most relevant explanatory variables and make prediction, obtaining a
moderately high accuracy. The results were confronted with the accuracy
of other Machine Learning algorithms, reaching the conclusion that the
MGM performed as well as other models. If the MGM gave a clear visual
representation of the interactions and the links between Bankrutpcy and
other variables, the other models tried to show which variables had the
most influence on the target. This represents two different approaches
with which making inference.

\hypertarget{bibliography}{%
\section{Bibliography}\label{bibliography}}

\url{https://www.kaggle.com/datasets/fedesoriano/company-bankruptcy-prediction}

Altman, E. (1968). Financial Ratios, Discriminant Analysis and the
Prediction of Corporate Bankruptcy. The Journal of Finance, 23(4),
p.589.

Altman, E. I., \& Hotchkiss, E. (2010). Corporate financial distress and
bankruptcy: Predict and avoid bankruptcy, analyze and invest in
distressed debt (Vol. 289). Hoboken, NJ: John Wiley \& Sons.

Avenhuis, J (2013) Testing the generalizability of the bankruptcy
prediction models of Altman, Ohlson and Zmijewski for Dutch listed
companies. Netherlands: University of Twente

Cochran, A. B. (1981). Small business mortality rates: a review of the
literature. Journal of Small Business Management, 19(4), 50--59.

Pretorius, M. (2009). Defining business decline, failure and turnaround:
a content analysis. Southern African Journal of Entrepreneurship and
Small Business Management, 2(1), 1--16.

Ohlson, J. (1980). Financial Ratios and the Probabilistic Prediction of
Bankruptcy. Journal of Accounting Research, 18(1), p.109

\end{document}
