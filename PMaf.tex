% Options for packages loaded elsewhere
\PassOptionsToPackage{unicode}{hyperref}
\PassOptionsToPackage{hyphens}{url}
%
\documentclass[
]{article}
\usepackage{amsmath,amssymb}
\usepackage{lmodern}
\usepackage{iftex}
\ifPDFTeX
  \usepackage[T1]{fontenc}
  \usepackage[utf8]{inputenc}
  \usepackage{textcomp} % provide euro and other symbols
\else % if luatex or xetex
  \usepackage{unicode-math}
  \defaultfontfeatures{Scale=MatchLowercase}
  \defaultfontfeatures[\rmfamily]{Ligatures=TeX,Scale=1}
\fi
% Use upquote if available, for straight quotes in verbatim environments
\IfFileExists{upquote.sty}{\usepackage{upquote}}{}
\IfFileExists{microtype.sty}{% use microtype if available
  \usepackage[]{microtype}
  \UseMicrotypeSet[protrusion]{basicmath} % disable protrusion for tt fonts
}{}
\makeatletter
\@ifundefined{KOMAClassName}{% if non-KOMA class
  \IfFileExists{parskip.sty}{%
    \usepackage{parskip}
  }{% else
    \setlength{\parindent}{0pt}
    \setlength{\parskip}{6pt plus 2pt minus 1pt}}
}{% if KOMA class
  \KOMAoptions{parskip=half}}
\makeatother
\usepackage{xcolor}
\IfFileExists{xurl.sty}{\usepackage{xurl}}{} % add URL line breaks if available
\IfFileExists{bookmark.sty}{\usepackage{bookmark}}{\usepackage{hyperref}}
\hypersetup{
  pdftitle={Probabilistic modeling for causal inference applied to African economic recessions},
  pdfauthor={Lorenzo Rossi},
  hidelinks,
  pdfcreator={LaTeX via pandoc}}
\urlstyle{same} % disable monospaced font for URLs
\usepackage[margin=1in]{geometry}
\usepackage{color}
\usepackage{fancyvrb}
\newcommand{\VerbBar}{|}
\newcommand{\VERB}{\Verb[commandchars=\\\{\}]}
\DefineVerbatimEnvironment{Highlighting}{Verbatim}{commandchars=\\\{\}}
% Add ',fontsize=\small' for more characters per line
\usepackage{framed}
\definecolor{shadecolor}{RGB}{248,248,248}
\newenvironment{Shaded}{\begin{snugshade}}{\end{snugshade}}
\newcommand{\AlertTok}[1]{\textcolor[rgb]{0.94,0.16,0.16}{#1}}
\newcommand{\AnnotationTok}[1]{\textcolor[rgb]{0.56,0.35,0.01}{\textbf{\textit{#1}}}}
\newcommand{\AttributeTok}[1]{\textcolor[rgb]{0.77,0.63,0.00}{#1}}
\newcommand{\BaseNTok}[1]{\textcolor[rgb]{0.00,0.00,0.81}{#1}}
\newcommand{\BuiltInTok}[1]{#1}
\newcommand{\CharTok}[1]{\textcolor[rgb]{0.31,0.60,0.02}{#1}}
\newcommand{\CommentTok}[1]{\textcolor[rgb]{0.56,0.35,0.01}{\textit{#1}}}
\newcommand{\CommentVarTok}[1]{\textcolor[rgb]{0.56,0.35,0.01}{\textbf{\textit{#1}}}}
\newcommand{\ConstantTok}[1]{\textcolor[rgb]{0.00,0.00,0.00}{#1}}
\newcommand{\ControlFlowTok}[1]{\textcolor[rgb]{0.13,0.29,0.53}{\textbf{#1}}}
\newcommand{\DataTypeTok}[1]{\textcolor[rgb]{0.13,0.29,0.53}{#1}}
\newcommand{\DecValTok}[1]{\textcolor[rgb]{0.00,0.00,0.81}{#1}}
\newcommand{\DocumentationTok}[1]{\textcolor[rgb]{0.56,0.35,0.01}{\textbf{\textit{#1}}}}
\newcommand{\ErrorTok}[1]{\textcolor[rgb]{0.64,0.00,0.00}{\textbf{#1}}}
\newcommand{\ExtensionTok}[1]{#1}
\newcommand{\FloatTok}[1]{\textcolor[rgb]{0.00,0.00,0.81}{#1}}
\newcommand{\FunctionTok}[1]{\textcolor[rgb]{0.00,0.00,0.00}{#1}}
\newcommand{\ImportTok}[1]{#1}
\newcommand{\InformationTok}[1]{\textcolor[rgb]{0.56,0.35,0.01}{\textbf{\textit{#1}}}}
\newcommand{\KeywordTok}[1]{\textcolor[rgb]{0.13,0.29,0.53}{\textbf{#1}}}
\newcommand{\NormalTok}[1]{#1}
\newcommand{\OperatorTok}[1]{\textcolor[rgb]{0.81,0.36,0.00}{\textbf{#1}}}
\newcommand{\OtherTok}[1]{\textcolor[rgb]{0.56,0.35,0.01}{#1}}
\newcommand{\PreprocessorTok}[1]{\textcolor[rgb]{0.56,0.35,0.01}{\textit{#1}}}
\newcommand{\RegionMarkerTok}[1]{#1}
\newcommand{\SpecialCharTok}[1]{\textcolor[rgb]{0.00,0.00,0.00}{#1}}
\newcommand{\SpecialStringTok}[1]{\textcolor[rgb]{0.31,0.60,0.02}{#1}}
\newcommand{\StringTok}[1]{\textcolor[rgb]{0.31,0.60,0.02}{#1}}
\newcommand{\VariableTok}[1]{\textcolor[rgb]{0.00,0.00,0.00}{#1}}
\newcommand{\VerbatimStringTok}[1]{\textcolor[rgb]{0.31,0.60,0.02}{#1}}
\newcommand{\WarningTok}[1]{\textcolor[rgb]{0.56,0.35,0.01}{\textbf{\textit{#1}}}}
\usepackage{graphicx}
\makeatletter
\def\maxwidth{\ifdim\Gin@nat@width>\linewidth\linewidth\else\Gin@nat@width\fi}
\def\maxheight{\ifdim\Gin@nat@height>\textheight\textheight\else\Gin@nat@height\fi}
\makeatother
% Scale images if necessary, so that they will not overflow the page
% margins by default, and it is still possible to overwrite the defaults
% using explicit options in \includegraphics[width, height, ...]{}
\setkeys{Gin}{width=\maxwidth,height=\maxheight,keepaspectratio}
% Set default figure placement to htbp
\makeatletter
\def\fps@figure{htbp}
\makeatother
\setlength{\emergencystretch}{3em} % prevent overfull lines
\providecommand{\tightlist}{%
  \setlength{\itemsep}{0pt}\setlength{\parskip}{0pt}}
\setcounter{secnumdepth}{-\maxdimen} % remove section numbering
\ifLuaTeX
  \usepackage{selnolig}  % disable illegal ligatures
\fi

\title{Probabilistic modeling for causal inference applied to African
economic recessions}
\author{Lorenzo Rossi}
\date{24/04/2022}

\begin{document}
\maketitle
\begin{abstract}
The paper
\end{abstract}

In 2003-2007, the developing world experienced an impressive economic
boom, growing at a rate of 7\% per year. The boom was fueled by a mix of
four ingredients prevailing in global markets: exceptional financing,
high commodity prices and, for a significant number of countries, large
flows of remittances. The first two conditions had coincided for the
last time in the 1970s, while the mix of the three had never been
experienced before. The rise of an alternative Asian engine, with China
at the center, is a fourth element, which has had a strong influence on
world trade and commodity prices.

ON BN

Bayesian networks are graphical models where nodes represent random
variables (the two terms are used interchangeably in this article) and
arrows represent probabilistic dependencies between them (Korb and
Nicholson 2004).

The main role of the network structure is to express the conditional
independence relationships among the variables in the model through
graphical separation

The graphical structure \(G = (V, A)\) of a Bayesian network is a
directed acyclic graph (DAG), where V is the node (or vertex ) set and A
is the arc (or edge) set. The DAG defines a factorization of the joint
probability distribution of \(V = {X1, X2, . . . , Xv}\), often called
the global probability distribution, into a set of local probability
distributions, one for each variable. The form of the factorization is
given by the Markov property of Bayesian networks (Korb and Nicholson
2004), which states that every random variable \(Xi\) directly depends
only on its parents \(ΠXi\) :

\#ON PC ALG Understanding cause-effect relationships between variables
is of primary interest in many fields of science. Usually, experimental
intervention is used to find these relationships. In many settings,
however, experiments are infeasible because of time, cost or ethical
constraints. More precisely, the causal effect \(C(V_1, V_6, x)\) of
\(V_1\) from \(V_1 = x\) on \(V_6\) is defined as\ldots{} If the causal
relationships are linear, the causal effect is independent of ˜x. Since
the causal structure was not identified uniquely in our example, we
cannot expect to get a unique number for the causal effect

Let \(G = (V, E)\) be a graph consisting of a set of vertices V and a
set of edges \(E ⊆ V × V\). The set of vertices that are adjacent to A
in graph G is defined as: \centering \(adj(A, G) = {B : (A, B) ∈ E\) or
\((B, A) ∈ E}\). B is called a collider if we have the v-structure:
\(A → B ← C\). Graph \(G\) is a Directed Acyclic Graph (DAG) if \(G\)
contains only directed edges and has no directed cycles. The skeleton of
a DAG G is the undirected graph obtained from G by ignoring the
direction of the edges. An equivalence class of DAGs is the set of DAGs
which have the same skeleton and the same v-structures. An equivalence
class of DAGs can be uniquely described by a completed partially
directed acyclic graph (CPDAG) which includes both directed and
undirected edges.

The first step of the algorithm consist of to build the skeleton graph:
for each pair of vertices u and v,

It assigns the directions to the edges in the skeleton, using the list
of conditional independencies found during the Step 1. The output is
completed partially directed acyclic graph, a mixed graph where there
are both directed and bidirected arcs.

The PC algorithm {[}2{]} (original-PC algorithm henceforward) has two
main steps. In the first step, it learns from data a skeleton graph,
which contains only undirected edges. In the second step, it orients the
undirected edges to form an equivalence class of DAGs. As the first step
of the PC algorithm contributes to most of the computational costs, we
only focus on the modification of this skeleton learning step in this
paper, and information about the edge orientation step can be found in
{[}2{]}. The theoretical foundation of the original-PC algorithm {[}2{]}
is that if there is no link (edge) between nodes X and Y , then there is
a set of vertices Z that either are neighbours of X or Y such that X and
Y are independent conditioning on Z. In other words, Z disconnects X and
Y

\begin{Shaded}
\begin{Highlighting}[]
\FunctionTok{summary}\NormalTok{(cars)}
\end{Highlighting}
\end{Shaded}

\begin{verbatim}
##      speed           dist       
##  Min.   : 4.0   Min.   :  2.00  
##  1st Qu.:12.0   1st Qu.: 26.00  
##  Median :15.0   Median : 36.00  
##  Mean   :15.4   Mean   : 42.98  
##  3rd Qu.:19.0   3rd Qu.: 56.00  
##  Max.   :25.0   Max.   :120.00
\end{verbatim}

\hypertarget{including-plots}{%
\subsection{Including Plots}\label{including-plots}}

You can also embed plots, for example:

\includegraphics{PMaf_files/figure-latex/pressure-1.pdf}

\#EConomic Background

A recession (fall in national income) will typically be characterised by
high unemployment, falling average incomes, increased inequality and
higher government borrowing. The impact of a recession depends on how
long it lasts and the depth of the fall in output.

The main costs of a recession will be:

Unemployment Fall in income -- shorter working week. Rise in poverty
Fall in asset prices (e.g.~fall in house prices/stock market) Increased
inequality and an increase in relative poverty Higher government
borrowing (less tax revenue) Permanently lost output. Firms go out of
business.

In particular, aggregate productivity can be affected by recessions in
two separate manners: i) by means of their impact on productivity within
each sector; ii) by inducing sectoral reallocations of input factors
across different sectors.4 The effect through sectoral reallocation is
not clear, since labor can move between various low- and
high-productivity sectors, with an ambiguous net effect on productivity.
Counter-cyclical reallocation takes place when input factor reallocation
during a downturn leads to less productive jobs being destroyed and
labor moving into more productive uses (Mortensen and Pissarides, 1994).
Pro-cyclical reallocation occurs when more productive industries are
disproportionately affected by recessions, for example due to credit
constraints (Barlevy, 2003).

Therefore, the extent to which recessions impact total factor
productivity (TFP) is ultimately an empirical question

\#\#ON AFRICA

Years Covered: 2000 to 2017.

Countries Covered: 27 African Countries Including: Morocco, South
Africa, Tanzania, Rwanda, Eswatini, Togo, Burkina Faso, Angola, Tunisia,
Nigeria, Kenya, Burundi, Benin, Namibia, Central African Republic,
Sudan, Gabon, Niger, Sierra Leone, Lesotho, Mauritania, Senegal,
Mauritius, Botswana, Cameroon, Zimbabwe and Mozambique.

Dramatic increases in international agricultural commodity prices began
in 2006 and peaked in July 2008. An equally remarkable and rapid decline
of those prices then ensued, accompanied by extreme volatility in those
prices. The trend in food prices lagged the rapid increases in other
commodity prices, including oil and metals, but accompanied those other
prices in the downward part of the cycle.

Greater regional integration offers more opportunities to climb the
technological ladder than do exports outside the continent, given that
intra-African exports are technologically more advanced, as indicated by
a larger share of medium- and high-technology manufactures
(International Trade Centre and UNCTAD, 2021; Saygili et al., 2018).
Moreover, intra-African trade consists of a higher share of processed
goods (41 per cent) than exports to the rest of the world (17 per cent),
and it is much more diversified in terms of traded products
(International Trade Centre and UNCTAD, 2021).

Obviously the collapse of the stock exchanges in the great finance
centres in May 2008 was also promptly transmitted to the stock exchanges
in the most important emerging countries. Consequently, not only were
the flows of portfolio and direct investments to the developing
countries significantly lower, but commercial bank credits and non-bank
financing were also reduced. On account of the world economic recession,
there was a sharp fall in the demand for goods and services from the
developing and emerging countries.

Dramatic increases in international agricultural commodity prices began
in 2006 and peaked in July 2008. An equally remarkable and rapid decline
of those prices then ensued, accompanied by extreme volatility in those
prices. The trend in food prices lagged the rapid increases in other
commodity prices, including oil and metals, but accompanied those other
prices in the downward part of the cycle. n developing countries, poor
market integration and border barriers may have limited pass-through of
these prices to the farmgate, but there was more rapid food price and
general inflation than occurred in many developed countries.

\end{document}
